% This is LLNCS.DEM the demonstration file of
% the LaTeX macro package from Springer-Verlag
% for Lecture Notes in Computer Science,
% version 2.4 for LaTeX2e as of 16. April 2010
%
\documentclass{llncs}
%
\usepackage{makeidx}  % allows for indexgeneration
\usepackage{url}
%
\begin{document}
%
\frontmatter          % for the preliminaries
%
\pagestyle{headings}  % switches on printing of running heads
%
\mainmatter              % start of the contributions
%
\title{Media Meets Semantic Web \newline How the BBC uses DBpedia and Linked Data \newline to Make Connections}
%
\titlerunning{Media Meets Semantic Web}  % abbreviated title (for running head), also used for the TOC unless \toctitle is used
%
\author{Andreas M\"{u}ller}
%
\authorrunning{Andreas M\"{u}ller} % abbreviated author list (for running head)
%
%%%% list of authors for the TOC (use if author list has to be modified)
\tocauthor{Andreas M\"{u}ller}
%
\institute{Technische Universit\"{a}t Berlin, 10623, Germany,\\
  \email{andreas.mueller.4@campus.tu-berlin.de},\\ WWW home page:
  \texttt{\url{http://www.user.tu-berlin.de/hpdesigner_20}}
}

\maketitle              % typeset the title of the contribution

\begin{abstract} % 70-150 words
This paper describes, how the BBC manages to better interlink different BBC domains by introducing DBpedia as a common vocabulary for every domain. Given the existing legacy systems, the BBC is already using, it is shown, how the new Semantic Web technology is integrated and used, to interlink documents and providing a better usability and user experience, allowing the user to browse different BBC domains by following a semantic thread.
\keywords{linked data, semantic web, bbc, dbpedia}
\end{abstract}
%
\section{Introduction}
%
The British Broadcasting Corporation (BBC) is one of the largest and the oldest Broadcasting Company in the world. Given the fact, that the BBC is producing online content since 1994 [TODO], they have a huge amount of online content today in text, audio and video format, splitted into different domains.
%
\section{Background}
%
...
%
\section{Solution}
%
...
%
\section{Evaluation}
%
...
%
\section{Related Work}
%
...
%
\section{Future Work \& conclusions}
%
...

%
% \subsection{Autonomous Systems}
%
% \begin{figure}
% \vspace{2.5cm}
% \caption{This is the caption of the figure displaying a white eagle and
% a white horse on a snow field}
% \end{figure}
%
% \paragraph{Notes and Comments.}
% The results in this section are a
% refined version of \cite{clar:eke};
% the minimality result of Proposition
% 14 was the first of its kind.

% \begin{table}
% \caption{This is the example table taken out of {\it The
% \TeX{}book,} p.\,246}
% \begin{center}
% \begin{tabular}{r@{\quad}rl}
% \hline
% \multicolumn{1}{l}{\rule{0pt}{12pt}
%                    Year}&\multicolumn{2}{l}{World population}\\[2pt]
% \hline\rule{0pt}{12pt}
% 8000 B.C.  &     5,000,000& \\
%   50 A.D.  &   200,000,000& \\
% 1650 A.D.  &   500,000,000& \\
% 1945 A.D.  & 2,300,000,000& \\
% 1980 A.D.  & 4,400,000,000& \\[2pt]
% \hline
% \end{tabular}
% \end{center}
% \end{table}
%

%
% ---- Bibliography ----
%
\begin{thebibliography}{5}
%
\bibitem {mmsw}
G.Kobilarov, T.Scott,Y .Raimond, S.Oliver, C.Sizemore, M.Smethurst, C.Bizer and R.Lee.:
Media meets semantic web - How the bbc uses dbpedia and linked data to make connections.
Lecture Notes in Computer Science (including subseries Lecture Notes in Artificial Intelligence and Lecture Notes in Bioinformatics)
5554LNCS:723–737, 2009.

\end{thebibliography}


%
% ---- Bibliography ----
%
\begin{thebibliography}{}
%
\bibitem[1980]{2clar:eke}
Clarke, F., Ekeland, I.:
Nonlinear oscillations and
boundary-value problems for Hamiltonian systems.
Arch. Rat. Mech. Anal. 78, 315--333 (1982)

\bibitem[1981]{2clar:eke:2}
Clarke, F., Ekeland, I.:
Solutions p\'{e}riodiques, du
p\'{e}riode donn\'{e}e, des \'{e}quations hamiltoniennes.
Note CRAS Paris 287, 1013--1015 (1978)

\bibitem[1982]{2mich:tar}
Michalek, R., Tarantello, G.:
Subharmonic solutions with prescribed minimal
period for nonautonomous Hamiltonian systems.
J. Diff. Eq. 72, 28--55 (1988)

\bibitem[1983]{2tar}
Tarantello, G.:
Subharmonic solutions for Hamiltonian
systems via a $\bbbz_{p}$ pseudoindex theory.
Annali di Matematica Pura (to appear)

\bibitem[1985]{2rab}
Rabinowitz, P.:
On subharmonic solutions of a Hamiltonian system.
Comm. Pure Appl. Math. 33, 609--633 (1980)

\end{thebibliography}

\end{document}
