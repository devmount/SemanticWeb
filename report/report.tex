% This is LLNCS.DEM the demonstration file of
% the LaTeX macro package from Springer-Verlag
% for Lecture Notes in Computer Science,
% version 2.4 for LaTeX2e as of 16. April 2010
%
\documentclass{llncs}
%
\usepackage{makeidx}  % allows for indexgeneration
\usepackage{url}
\usepackage{graphicx}
%
\begin{document}
%
\frontmatter          % for the preliminaries
%
\pagestyle{headings}  % switches on printing of running heads
%
\mainmatter              % start of the contributions
%
\title{Media Meets Semantic Web:\newline Interlinking of BBC microsites\newline with DBpedia and Linked Data technologies}
%
\titlerunning{Media Meets Semantic Web}  % abbreviated title (for running head), also used for the TOC unless \toctitle is used
%
\author{Andreas M\"{u}ller}
%
\authorrunning{Andreas M\"{u}ller} % abbreviated author list (for running head)
%
%%%% list of authors for the TOC (use if author list has to be modified)
\tocauthor{Andreas M\"{u}ller}
%
\institute{Technische Universit\"{a}t Berlin, 10623, Germany,\\
  \email{andreas.mueller.4@campus.tu-berlin.de},\\ WWW home page:
  \texttt{\url{http://www.user.tu-berlin.de/hpdesigner_20}}
}

\maketitle              % typeset the title of the contribution

\begin{abstract} % 70-150 words
This paper describes, how the BBC managed to better interlink different BBC domains by introducing DBpedia as a common vocabulary for every domain. Given the existing legacy systems, the BBC was already using, it is shown, how the new Semantic Web technology was integrated and used, to interlink documents and providing a better usability and user experience, allowing the user to browse different BBC domains by following a semantic thread and getting cross-domain information.
\keywords{linked data, semantic web, bbc, dbpedia}
\end{abstract}

\section{Introduction}

The British Broadcasting Corporation (BBC) is the oldest and still one of the largest Broadcasting Companies in the world. Given the fact, that the BBC is producing online content since 1994 \cite{st1994}, they have a huge amount of online media content today in text, audio and video format. To make the data accessible, it was categorized and organized in different domains, i.e. news, sport, weather etc. Figure \ref{fig:domains2009} and \ref{fig:domains2017} present these domains the way they were displayed on the BBC website in the years 2009 and 2017. Each domain became a separate microsite with its own content, vocabulary and datasets.
\begin{figure}[!ht]
  \includegraphics[width=\textwidth]{images/bbc_domains_2009}
  \caption{BBC microsites in June, 2009}
  \label{fig:domains2009}
\end{figure}
\begin{figure}[!ht]
  \includegraphics[width=\textwidth]{images/bbc_domains_2017}
  \caption{BBC microsites in June, 2017}
  \label{fig:domains2017}
\end{figure}

This separation of content created a clear structure where the user instantly knew where to navigate to when he searched for content of a particular domain. But it also came with a big disadvantage: It was neither possible to find everything, the BBC has published to a given subject nor to navigate between different BBC domains following a semantic thread (i.e. on a page about a musician was no possibility to see all programmes that played this artist). The reason for this lack of interoperability was the missing interlinking between the different microsites. Without this interlinking, the real potential of the available data was not used.

\vspace{15mm}

To make the BBC website more coherent and more useful, G.Kobilarov et al. proposed a solution \cite{mmsw} with the following objectives:
\begin{enumerate}
  \item Build better connections and interlinking of existing systems
  \item Reducing impact on existing systems while adding new services to maximize interlinking of domains
\end{enumerate}

The next section will give a short overview of the existing legacy systems of the BBC website.

\section{Background: Legacy Systems}

One of the systems, the BBC was already using to automate content relations, was a legacy auto-categorization system called \textit{CIS}. With this system it was possible, to categorize programmes by their textual description, that contained meta data like brands, locations, people and subjects. Despite being used for the programmes domain only, it was not possible to cover every single entity that might have been of interest with CIS. This system also held no information about relations between different terms; i.e. the terms "Bejing" and "Bejing Olympics" obviously are related by location, but in the CIS system these were seen as just two independet terms. Additionally the system only relied on internal identifiers, so linking to non-BBC data was not possible either.

G.Kobilarov et al. pointed out, that CIS could only be used to interlink between different domains if there were mappings between the various vocabularies of each domain. In this case, it would be possible to further developing them independently \cite{mmsw}.

\section{Solution: Integration of DBpedia}

To solve the problem of the need for a common set of web identifiers for all domains, a well-known, crowd-sourced extraction framework was chosen: \textit{DBpedia}\footnote{The first public DBpedia datasets were released 2007. Its purpose is to extract structured content from the data of Wikipedia using Semantic Web and Linked Data technologies \cite{dbpedia}.}.

\begin{figure}[!h]
  \centering
  \includegraphics[width=.75\textwidth]{images/dbpedia_dark}
  \caption{BBC microsites connected by DBpedia}
  \label{fig:dbpedia}
\end{figure}

To interlink all domains (as to be seen in figure \ref{fig:dbpedia}), DBpedia serves as a common vocabulary. This works in two steps. At first, a DBpedia label lookup is performed for a given CIS term. Therefore the most likely matches to a given term and possible DBpedia resources are found and ranked by their relevance\footnote{The relevance here is calculated by the number of backlinks.}. After that, the best match is found by context-based disambiguation. All relevent matches are disambiguated by clustering them and finding an according context in DBpedia. The term "apple" i.e. is simply a fruit for itself, but in the context of "Microsoft" and "Google" it becomes "Apple Inc.".

With the use of the same vocabulary, the interlinking of text documents (i.e. news or documentation articles) became possible. Therefore the text body of a BBC document URI was parsed to extract the main entities using \textit{Named-entity Recognition} (NER)\footnote{Named-entity Recognition is a part of information extraction, that finds and categorizes real-world objects in text.}. These main entities then are matched by an algorithm to possible DBpedia resources and ranked by contextual disambiguation, like described in the previous paragraph. This creates a mapping of extracted terms to possible counterparts in DBpedia. Finally the DBpedia resources are filtered in a way that only resources that correspond to "people" or "companies" are kept. This is achieved by the already existing predicates of the terms provided by CIS. This process of interlinking documents is realized by a system called \textit{Muddy Boots} and is presented in figure \ref{fig:muddy}.

\begin{figure}[!h]
  \includegraphics[width=\textwidth]{images/muddy_boots_dark}
  \caption{Muddy Boots: Identify main actors in a piece of text content}
  \label{fig:muddy}
\end{figure}

\section{Evaluation}

According to G.Kobilarov et al. \cite{mmsw} the precision of the interlinking results previously described are very good, with a precision of 86\% for Brands and more than 90\% for locations, names and subjects (see table \ref{tab:evaluation}).
\begin{table}
\centering
\begin{tabular}{lcccc}
  \hline
  \textbf{ } & \textbf{Total} & \textbf{Linked} & \textbf{Precision} & \textbf{Recall} \\
  \hline
  Brand & 6,630 & 1,267 (19\%) & 86\% & 41\% \\
  Location & 55,943 & 11,316 (20\%) & 99\% & 77\% \\
  Name & 73,442 & 22,341 (30\%) & 92\% & 67\% \\
  Subject & 11,231 & 6,822 (61\%) & 92\% & 75\% \\
  \hline
\end{tabular}
\caption{CIS / DBpedia Interlinking Results of G.Kobilarov et al. \cite{mmsw}}
\label{tab:evaluation}
\end{table}

The reason for the low percentage values (only up to 30\%) in the "Linked" column for brands, locations and names is, that there is simply no particular Wikipedia article for many terms in these categories.

\section{Related Work}

Making current data accessible by linking information is not the only issue the BBC is facing. One problem i.e. is the accessability of a media archive. Raimond et al. proposed "a system to process the existing audio and text and automatically annotate programmes within the archive with Linked Data web identifiers" \cite{raimond} in 2014. And as automated data always fails for a small portion of generated content, they even developed a crowdsourcing mechanism to create the possibility for users to take part in correction or addition.

Furthermore the BBC has evolved a rich collection of learning resources. Mikroyannidi et al. proposed a system "that employs semantic web technologies to organize the available learning resources" in 2016 \cite{education}. They describe how such a data model and architecture could look like.

\section{Future Work \& conclusions}

G.Kobilarov et al. \cite{mmsw} showed, that it is possible to develop a smart interlinking to internal and external resources of previously seperated content domains to improve the user experience. Users now can access information from different domains much easier following more meaningful navigation paths.

It is conceivable, that an approach based on \textit{Natural Language Processing} and \textit{Neural Networks} could achieve an even higher score in precision and recall when linking to interal or external resources, based on the fact that the BBC has more than enough data to train language models that represent their articles.

The improvements to systems like the BBC website described in the previous sections show, that the process of improving user experience and at the same time handling the legacy systems will always be a current issue.


\vspace{20mm}

%
% ---- Bibliography ----
%
\begin{thebibliography}{5}
%
\bibitem {mmsw}
  G.Kobilarov, T.Scott,Y .Raimond, S.Oliver, C.Sizemore, M.Smethurst, C.Bizer and R.Lee.:
  \textit{Media meets semantic web - How the bbc uses dbpedia and linked data to make connections}.
  Lecture Notes in Computer Science (including subseries Lecture Notes in Artificial Intelligence and Lecture Notes in Bioinformatics)
  5554LNCS:723–737, 2009.
%
\bibitem {st1994}
  The Sunday Times:
  \textit{THE BBC is launching an on-line service}. 17 April 1994.
  Quoted in Connor, Alan (25 December 2007). "The WWW Info-Rainforest". \textit{BBC Internet Blog}. BBC.
%
\bibitem {raimond}
  Yves Raimond, Tristan Ferne, Michael Smethurst, Gareth Adams:
  \textit{The BBC World Service Archive prototype}.
  Web Semantics: Science, Services and Agents on the World Wide Web
  Semantic Web Challenge 2013:2-9, 2014
%
\bibitem {dbpedia}
  J.Lehmann, R.Isele, M.Jakob, A.Jentzsch, D.Kontokostas, P.N.Mendes, S.Hellmann, M.Morsey, P.van Kleef, S.Auer, C.Bizer:
  \textit{DBpedia – A large-scale, multilingual knowledge base extracted from Wikipedia}.
  10.3233/SW-140134
  Semantic Web, vol. 6, no. 2, pp. 167--195, 2015
%
\bibitem {education}
  Eleni Mikroyannidi, Dong Liu, Robert Lee:
  \textit{Use of Semantic Web Technologies in the Architecture of the BBC Education Online Pages}.
  Open Data for Education: Linked, Shared, and Reusable Data for Teaching and Learning, pp. 67--85, 2016
%
\end{thebibliography}

\end{document}
